\documentclass[11pt,letterpaper,boxed]{pset}

\usepackage[margin=0.75in]{geometry}
\usepackage{ulem}

\name{Name: \rule{2.5cm}{0.15mm}}
\assignment{Box \# \rule{1.5cm}{0.15mm}}
\class{MATH065 HW2}

\begin{document}

    \problemlist{MATH065 HW2}
    \begin{center}
        6.2: 30, 36, 40, 42, 44, 58 \\
        6.3: 6, 8
    \end{center}
    
    %------------------------- 6.2.30 -----------------------
    
    \begin{problem}[6.2 \#30]
    Let $\mathscr{B}$ be a set of vectors in a vector space $V$ with
    the property that every vector in $V$ can be written uniquely as a linear combination of the vectors in $\mathscr{B}$. Prove that $\mathscr{B}$ is a basis for $V$.
    \end{problem}
    \newpage
    
    %------------------------- 6.2.36 -----------------------
    
    \begin{problem}[6.2 \#36]
    For Exercises 34-39, find the dimension of the vector space $V$ and give a basis for $V$.
    \[ V=\{p(x) \in \mathscr{P}_2: xp'(x)=p(x) \} \]
    \end{problem}
    \newpage
    
    %------------------------- 6.2.40 -----------------------
    
    \begin{problem}[6.2 \#40]
    Find a formula for the dimension of the vector space of symmetric $n$x$n$ matrices.
    \end{problem}
    \newpage
    
    %------------------------- 6.2.42 -----------------------
    
    \begin{problem}[6.2 \#42]
    Let $U$ and $V$ be subspaces of a finite-dimensional vector space $V$. Prove \textbf{\textit{Grassmann's Identity:}}
    \[ dim(U+W)=dimU+dimW-dim(U \cap W) \]
    [\textit{Hint:} The subspace $U+W$ is defined in Exercise 48 of Section 6.1. Let $\mathscr{B}= \{\Vec{v}_1,...,\Vec{v}_k\}]$ be a basis for $U \cap W$. Extend $\mathscr{B}$ to a basis of $\mathscr{C}$ of $U$ and a basis $\mathscr{D}$ of $W$. Prove that $\mathscr{C} \cup \mathscr{D}$ is a basis for for $U+W$.]
    \end{problem}
    \newpage
    
    %------------------------- 6.2.44 -----------------------
    
    \begin{problem}[6.2 \#44]
    Prove that the vector space $\mathscr{P}$ is infinite-dimensional. [\textit{Hint:} Suppose it has a finite basis. Show that there is some polynomial that is not a linear combination of this basis.]
    \end{problem}
    \newpage
    
    %------------------------- 6.2.58 -----------------------
    
    \begin{problem}[6.2 \#58]
    Let $\{\textbf{v}_1,...,\textbf{v}_n\}$ be a basis for a vector space $V$. Prove that 
    \[ \{ \textbf{v}_1, \textbf{v}_1+\textbf{v}_2, \textbf{v}_1+\textbf{v}_2+\textbf{v}_3,...,\textbf{v}_1+...+\textbf{v}_n \} \]
    is also a basis for $V$.
    \end{problem}
    \newpage
    
    %------------------------- 6.3.6 -----------------------
    
    \begin{problem}[6.3 \#6]
    In Exercises 5-8, follow the instructions for Exercises 1-4 using p(x) instead of \textbf{x}. In Exercises 1-4: (a) Find the coordinate vectors [$\textbf{x}$] $_{\mathscr{B}}$ and [$\textbf{x}$] $_{\mathscr{C}}$ of \textbf{x} with respect to the bases $\mathscr{B}$ and $\mathscr{C}$, respectively. (b) Find the change-of-basis $P_{\mathscr{C}\leftarrow\mathscr{B}}$ from $\mathscr{B}$ to $\mathscr{C}$. (c) Use your answer to part (b) to compute [$\textbf{x}$] $_{\mathscr{C}}$, and compare your answer with the one found in part (a). (d) Find the change-of-basis matrix $P_{\mathscr{B}\leftarrow\mathscr{C}}$ from $\mathscr{C}$ to $\mathscr{B}$. (e) Use your answers to parts (c) and (d) to compute [$\textbf{x}$] $_{\mathscr{B}}$, and compare your answer with the one found in part (a).
    
    \[ p(x)=1+3x, \mathscr{B}=\{1+x,1-x\}, \mathscr{C}=\{2x,4\} \in \mathscr{P}_1 \]
    
    \end{problem}
    \newpage
    
    %------------------------- 6.3.8 -----------------------
    
    \begin{problem}[6.3 \#8]
    In Exercises 5-8, follow the instructions for Exercises 1-4 using p(x) instead of \textbf{x}. In Exercises 1-4: (a) Find the coordinate vectors [$\textbf{x}$] $_{\mathscr{B}}$ and [$\textbf{x}$] $_{\mathscr{C}}$ of \textbf{x} with respect to the bases $\mathscr{B}$ and $\mathscr{C}$, respectively. (b) Find the change-of-basis $P_{\mathscr{C}\leftarrow\mathscr{B}}$ from $\mathscr{B}$ to $\mathscr{C}$. (c) Use your answer to part (b) to compute [$\textbf{x}$] $_{\mathscr{C}}$, and compare your answer with the one found in part (a). (d) Find the change-of-basis matrix $P_{\mathscr{B}\leftarrow\mathscr{C}}$ from $\mathscr{C}$ to $\mathscr{B}$. (e) Use your answers to parts (c) and (d) to compute [$\textbf{x}$] $_{\mathscr{B}}$, and compare your answer with the one found in part (a).
    
    \[ p(x)=4-2x-x^2, \mathscr{B}=\{x,1+x^2,x+x^2\}, \mathscr{C}=\{1,1+x,x^2\} \in \mathscr{P}_2 \]
    
    \end{problem}
    \newpage
    
\end{document}