\documentclass[11pt,letterpaper,boxed]{pset}

\usepackage[margin=0.75in]{geometry}
\usepackage{ulem}

\name{Name: \rule{2.5cm}{0.15mm}}
\assignment{Box \# \rule{1.5cm}{0.15mm}}
\class{MATH065 HW6}

\begin{document}

    \problemlist{MATH065 HW6}
    
    \begin{problem} [Exercise 1.]
    Find the general solution of $\mathbf{x} \, ' = A  \mathbf{x}$ for the matrix $A = \begin{bmatrix} 2 & 1 \\ -3 & 6 \end{bmatrix}$. Express your answer in the form $\mathbf{x}(t) = \Psi(t) \mathbf{c}$ where $\Psi(t)$ is a fundamental matrix.
    \end{problem}
    \newpage
    
    
    \begin{problem} [Exercise 2.]
    Find the general solution of $\mathbf{x} \, ' = A  \mathbf{x}$ for the matrix $A = \begin{bmatrix} 2 & 2 & 1 \\ 1 & 3 & 1 \\ 1 & 2 &2 \end{bmatrix}.$
    \end{problem}
    \newpage
    
    
    \begin{problem} [Exercise 3.]
    Suppose  $\lambda \in \mathbb{C}$  and  $\mathbf{v} \in \mathbb{C}^n$. Determine the real and imaginary parts of the  function
    
    \[  \mathbf{x}(t) = e^{\lambda t} \mathbf{v}.  \]
    
    For consistency in grading, assume $\lambda = \alpha + i \beta$   ($\alpha,\beta \in \mathbb{R}$) and $\mathbf{v} = \mathbf{p} + i \mathbf{q}$  ($\mathbf{p},\mathbf{q} \in \mathbb{R}^n$). 
    \end{problem}
    \newpage
    
    
    \begin{problem} [Exercise 4.]
    Find the real-valued general solution for the system
    \begin{equation}
    \mathbf{x} \, ' = \begin{bmatrix} 2 & -1 \\ 1 & 2 \end{bmatrix} \mathbf{x}.
     \end{equation}
    Express your answer in the form $\mathbf{x}(t) = \Psi(t) \mathbf{c}$ where $\Psi(t)$ is a fundamental matrix.
    \end{problem}
    \newpage
    
    
    
    \begin{problem} [Exercise 5.]
    Solve the initial-value problem
    \begin{align}
    \dot{x} &= -y \\
    \dot{y} &= x 
    \end{align}
    with $x(0)=1, y(0)=-1$ by first identifying the matrix $A$ and then using the eigendata. Express your answer in the form $\mathbf{x}(t) = \Psi(t) \mathbf{x}_0$ where $\Psi(t)$ is a fundamental matrix  and $\mathbf{x}_0 $ is the initial state vector.
    \end{problem}
    \newpage
    
    \begin{problem} [Exercise 6.]
    (Math 45 version of Exercise 5). Solve the initial-value problem
    \begin{align}
        \dot{x} &= -y \\
        \dot{y} &= x 
    \end{align}
    
    with $x(0)=1, y(0)=-1$ by first differentiating $\dot{x} = -y$ and using Math 45 knowledge. 
    \end{problem}
    \newpage
    
    
    
    \begin{problem} [Exercise 7.]
    (Math 60 Version of Exercise 5). Calculate the flow line $\mathbf{x}(t)$ for the given vector field $\mathbf{F}$ that passes through the indicated point at the specified value of $t$. 
    \[ \mathbf{F}(x,y)=-y \mathbf{i} + x \mathbf{j}; \mathbf{x}(0)=(1,-1).   \]
    Sketch the vector field and the flow line.
    \end{problem}
    \newpage
    
    
    \begin{problem} [Exercise 8.]
    (More Math 60 insights). Let $E(x,y) = x^2 + y^2$ and suppose $\mathbf{x}(t) = (x(t),y(t))$ is a flow line of  $\mathbf{F}(x,y)=-y \mathbf{i} + x \mathbf{j} $ (i.e., solution of the system $\dot{x}  =   - y  \textrm{ and } \dot{y} =   x $,  in exercise 5 and 6). Use the chain rule to calculate 
    \[  \frac{d}{dt} E(\mathbf{x}(t)).  \]
    Why does this imply the flow lines are circles? 
    \end{problem}
    \newpage

\end{document}